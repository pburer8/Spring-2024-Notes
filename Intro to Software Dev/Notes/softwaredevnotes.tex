\documentclass[12pt]{article}

\usepackage{amsmath, amsfonts, amssymb, setspace, blindtext, hyperref, tikz, graphicx} \doublespacing

\hypersetup{colorlinks=true,linkcolor=blue,filecolor=magenta,urlcolor=cyan,}
\title{Introduction to Software Dev Notes}

\author{Marie Burer}

\date{16-1-24}

\begin{document}

\maketitle
\newpage
\setlength{\parindent}{0pt}
\tableofcontents

\newpage

\section{Overview}

Tools:
\begin{itemize}
    \item Warp
    \item Eclipse
    \item Git
\end{itemize}
What is Warp?

\begin{itemize}
    \item Existing code base
    \begin{itemize}
        \item Not huge, but bigger than other class projects
        \item Useful because it reflects real world, you won't join a team and write code from scratch
    \end{itemize}

    \item Written by Professor Goddard for a research project called WARP
    \begin{itemize}
        \item Initially written in Swift, then rewritten in Java for the course
        \item Primary purpose is to develop programs for network communication in wireless sensor networks
    \end{itemize}
\end{itemize}
Why Warp?

\begin{itemize}
    \item Get used to reading and modifying others' codes
    \item Learn to refactor code
    \item Rich opportunity to add:
    \begin{itemize}
        \item documentation
        \item pre- and post-conditions
        \item testing
        \item new features
    \end{itemize}
    \item Learn from Professor Goddard's experiences and mistakes.
\end{itemize}

\section{Software Development Process Models}

Beginner's Model: Code and Fix
\begin{itemize}
    \item Conceptual Development
    \item Code and Fix
    \item Release Product
\end{itemize}

Software Development Life Cycle
\begin{itemize}
    \item Conception
    \item Requirements gathering
    \item Design
    \item Coding and debugging
    \item Testing
    \item Release
    \item Maintenance
    \item Retirement
\end{itemize}

Software Development Model Categories
\begin{itemize}
    \item Plan driven models:
    \begin{itemize}
        \item Strict methodology
        \item Clearly defined phases
        \item Heavy weight
        \item Works best for large contracts
    \end{itemize}
    \item Agile development models:
    \begin{itemize}
        \item incremental and cyclic
        \item small, frequent releases
        \item less documentation
        \item Works well for start ups
    \end{itemize}
\end{itemize}

Waterfall Process Model
\begin{itemize}
    \item Conceptional Development
    \item Requirements Analysis
    \item Architectural Design
    \item Detailed Design
    \item Code and Debug
    \item System Testing 
    \item Release and Maintenance
\end{itemize}

\newpage
\section{Code Construction}

Who is the code for?
\begin{itemize}
    \item The computer
    
    Must fulfill requirements and implement the design

    \item You and other programmers
    
    Must be readable and easy to understand 
\end{itemize}

\newpage
\section{Debugging}

Testing finds errors, debugging repairs them

\section{Stuff I Didn't write down}

\section{Object Oriented Design Principles}

\subsection{Liskov Substitution Principle}

The key of OCP: Abstraction and Polymorphism

Implemented by inheritance, how do we measure quality?

Example: Birds can fly, penguins cannot fly.

Throwing an error does not model "penguins cannot fly", it models "try, error"

Design by contract:

Advertised behavior of an object, requirements and promises

Derived class services should require no more, promise no less.

Is-A relationship must refer to the behavior of the class

Replaceability: Any code which can legally call a class's methods must be able to substitute any subclass without modification

\subsection{Dependency Inversion Principle}

High-level should not depend on low-level, both should depend on abstractions

Design to an inheritance not an implementation

Abstract classes/interfaces are not subject to change

\subsection{The Founding Principles}

Closely related, violating one means violating others

Keep them in mind at all times
\end{document}