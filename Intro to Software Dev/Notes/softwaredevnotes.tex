\documentclass[12pt]{article}

\usepackage{amsmath, amsfonts, amssymb, setspace, blindtext, hyperref, tikz, graphicx} \doublespacing

\hypersetup{colorlinks=true,linkcolor=blue,filecolor=magenta,urlcolor=cyan,}
\title{Introduction to Software Dev Notes}

\author{Marie Burer}

\date{16-1-24}

\begin{document}

\maketitle
\newpage
\setlength{\parindent}{0pt}
\tableofcontents

\newpage

\section{Overview}

Tools:
\begin{itemize}
    \item Warp
    \item Eclipse
    \item Git
\end{itemize}
What is Warp?

\begin{itemize}
    \item Existing code base
    \begin{itemize}
        \item Not huge, but bigger than other class projects
        \item Useful because it reflects real world, you won't join a team and write code from scratch
    \end{itemize}

    \item Written by Professor Goddard for a research project called WARP
    \begin{itemize}
        \item Initially written in Swift, then rewritten in Java for the course
        \item Primary purpose is to develop programs for network communication in wireless sensor networks
    \end{itemize}
\end{itemize}
Why Warp?

\begin{itemize}
    \item Get used to reading and modifying others' codes
    \item Learn to refactor code
    \item Rich opportunity to add:
    \begin{itemize}
        \item documentation
        \item pre- and post-conditions
        \item testing
        \item new features
    \end{itemize}
    \item Learn from Professor Goddard's experiences and mistakes.
\end{itemize}

\end{document}