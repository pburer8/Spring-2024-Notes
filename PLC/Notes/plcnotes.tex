\documentclass[12pt]{article}

\usepackage{amsmath, amsfonts, amssymb, setspace, blindtext, hyperref, tikz, graphicx} \doublespacing

\hypersetup{colorlinks=true,linkcolor=blue,filecolor=magenta,urlcolor=cyan,}
\title{Programming Language Concepts Notes}

\author{Marie Burer}

\date{18-1-24}

\begin{document}

\maketitle
\newpage
\setlength{\parindent}{0pt}

\tableofcontents

\newpage

\section{Introduction to Haskell}

Functional programming: Functions are central

Garbage collection came first with FP, and recursion is used over loops.

Avoid side effects

Side effects include changing some hidden state. Functions should return the same values when given the same input ideally.

Haskell enforces purity, no side effects

Haskell is statically typed, lazy semantics

FP also used to make compilers and interpreters

Learning Haskell is good for you, a new paradigm and way to think

\section{Higher-order functions in Haskell}

In FP, functions are first-class
\begin{itemize}
    \item Can be outputs
    \item Can be inputs
\end{itemize}

To call a function f with argument a, f a

(f . g) x = f (g x)

Consider:

f :: (a $\to$ b) $\to$ (a,a) $\to$ (b,b)

f g (x,y) = (g x, g y)

Anonymous function:

$\backslash x \to x * x$

In definitions:

$id = \backslash x \to x$

As arguments to higher order functions:

applyTwice($\backslash x \to x * x$)

\subsection{Partial Applications and Sections}

Suppose we have f :: Int $\to$ Int $\to$ Int

Then we can write f 3 :: Int $\to$ Int

Int $\to$ Int $\to$ Int is the same as Int $\to$ (Int $\to$ Int)

If we have something like $(+ 10)$, that is a section

\subsection{List Combinators}

Consider map :: (a $\to$ b) $\to$ ([a] $\to$ [b])

f :: a $\to$ b

map f xs applies f to all elements of xs

filter :: (a $\to$ Bool) $\to$ [a] $\to$ [a]

filter p xs keeps every element of xs where p x is True

zipWith :: (a $\to$ b $\to$ c) $\to$ [a] $\to$ [b] $\to$ [c]

zipWith f [x1, ..., xn][y1, ..., y(n+k)] = [f x1 y1, ..., f xn yn]


\end{document}