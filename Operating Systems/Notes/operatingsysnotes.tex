\documentclass[12pt]{article}

\usepackage{amsmath, amsfonts, amssymb, setspace, blindtext, hyperref, tikz, graphicx} \doublespacing

\hypersetup{colorlinks=true,linkcolor=blue,filecolor=magenta,urlcolor=cyan,}
\title{Operating Systems Notes}

\author{Marie Burer}

\date{16-1-24}

\begin{document}

\maketitle
\newpage
\setlength{\parindent}{0pt}

\tableofcontents

\newpage

\section*{Course Outline}

Information:
\begin{itemize}
    \item CS:3620 Operating Systems
    \item Instructor: Guanpeng Li
    \item Location: 110 MacLean Hall
    \item Hours: Tuesday/Thursday, 8:00 AM - 9:15 AM
    \item Prerequesite: CS:2210, CS:2230, and CS:2630 with a minimum grade of C-
    \item TA: A K M Muhitul Islam
\end{itemize}

Modality:
\begin{itemize}
    \item Lectures in person
    \item Office Hours: Tuesday/Thursday, 2:30 PM - 4:30 PM
    \item Location: IATL 340
\end{itemize}

Grading:
\begin{itemize}
    \item Homework: 40\%
    \begin{itemize}
        \item 6-8 assignments
        \item Programming tasks
    \end{itemize}

    \item In-class activities: 20\%
    \begin{itemize}
        \item Quizzes:
        \begin{itemize}
            \item Multiple-choice
            \item Close book
        \end{itemize}
        \item Exercises:
        \begin{itemize}
            \item Open questions
            \item Individual or group
            \item BYO laptop
        \end{itemize}
    \end{itemize}
    
    \item Final Exam: 40\%
    \begin{itemize}
        \item Technical interview
        \item Written exam
    \end{itemize}
\end{itemize}

Policy:
\begin{itemize}
    \item 3 days late policy
    \item Cannot make up quizzes or exercises
    \item Can discuss homework with colleagues
    \item Cannot share code
    \item Cannot copy from internet
    \item If you copy code from StackOverflow, credit it 
    \item When in doubt, ask the instructor/TA
\end{itemize}

\newpage

\section{Introduction to OS}
What is an operating system?

An operating system (OS) is system software that manages computer hardware and software resources and provides common services for computer programs. (Wikipedia)
\begin{itemize}
    \item Definition has changed over years
    \begin{itemize}
        \item Originally, very bare bones
        \item Evolved to include more
    \end{itemize}
    \item Operating System (OS)
    \begin{itemize}
        \item Interface between the user and the architecture
        \item Implements a virtual machine that is easier to program than raw hardware
    \end{itemize}
    \item Middleware between user programs and system hardware
    \item Manages hardware: CPU, main memory, I/O devices
\end{itemize}
OS: Traditional View:
\begin{itemize}
    \item Interface between user and architecture
    \begin{itemize}
        \item Hides architectural details
    \end{itemize}
    \item Implements Virtual machine:
    \begin{itemize}
        \item Easier to program
    \end{itemize}
    \item Illusionist
    \begin{itemize}
        \item Bigger, faster, reliable
    \end{itemize}
    \item Government
    \begin{itemize}
        \item Divides resources
        \item "Taxes" = overhead
    \end{itemize}
\end{itemize}
New Developments in OS:
\begin{itemize}
    \item Operating systems: active field of research
    \begin{itemize}
        \item Demands on OS's growing
        \item New application spaces (Web, Grid, Cloud)
        \item Rapidly evolving hardware
    \end{itemize}
    \item Open-source operating systems
    \begin{itemize}
        \item Linux etc.
        \item You can contribute to and develop OS's
        \item Excellent research platform
    \end{itemize}
\end{itemize}
OS: Salient Features
\begin{itemize}
    \item Services: The OS provides standard services which the hardware implements
    \item Coordination: The OS coordinates multiple applications and users to achieve fairness and efficiency
    \item Design an OS so that the machine is convenient to use and efficient
\end{itemize}
Why study OS?
\begin{itemize}
    \item Abstraction: How to get the OS to give users an illusion of infinite resources
    \item System Design: How to make tradeoffs between performance, convenience, simplicity, and functionality
    \item Basic Understanding: The OS provides the services that allow application programs to work at all
    \item System Intersection Point: The OS is the point where hardware and software meet
\end{itemize}

Not many operating systems are under development, so you are unlikely to get a job building an OS. However, understanding operating systems will enable you to use your computer more efficiently \newline
Build Large Computer Systems
\begin{itemize}
    \item OS as an example of large system design
    \item Goals: 
\end{itemize}
\newpage
\section{OS Structure and Architecture}

Modern OS Functionality:
\begin{itemize}
    \item Concurrency
    \item I/O Devices
    \item Memory management
    \item Files
    \item Distributed systems and networks
\end{itemize}

OS Principles:
\begin{itemize}
    \item OS as juggler
    \item OS as government
    \item OS as complex system
    \item OS as history teacher
\end{itemize}

Protection:
\begin{itemize}
    \item Kernel mode vs User Mode
    
    To protect the system from aberrant users, some instructions are restricted to use only by the OS

    Users may not
    \begin{itemize}
        \item address I/O directly
        \item use instructions that manipulate the state of memory
        \item set the mode bits that determine user or kernel mode
        \item disable and enable interrupts
        \item halt the machine
    \end{itemize}

    In kernel mode, the OS can do all of these things

    \item Hardware needs to support at least both kernel and user
\end{itemize}

Crossing Protection Boundaries:

System call: OS procedure that executes privileged instructions
\begin{itemize}
    \item Causes a trap
    \item The trap handler uses the parameter to jump
    \item The handler saves caller's state so it can restore control
    \item The architecture must permit this
\end{itemize}

Memory Protection:
\begin{itemize}
    \item Architecture must provide support to the OS to protect user programs and itself
    \item The simplest technique is to use base and limit registers
    \item Instantiated by the OS before starting a program
    \item The CPU checks user reference ensuring it falls between base and limit values
\end{itemize}

Memory Hierarcy:

Higher = small, fast, more \$, less latency
\begin{itemize}
    \item registers (1 cycle)
    \item L1 (2 cycle)
    \item L2 (7 cycle)
    \item RAM (100 cycle)
    \item Disk (40,000,000 cycle)
    \item Network (200,000,000+ cycle)
\end{itemize}

Process Layout in Memory
\begin{itemize}
    \item Stack
    \item Gap 
    \item Data 
    \item Text
\end{itemize}

Caches
\begin{itemize}
    \item Access to main memory is expensive
    \item Caches are small, fast, inexpensive
    \begin{itemize}
        \item Different sizes and locations:
        \begin{itemize}
            \item L1: on-chip, smallish
            \item L2: next to chip, larger
            \item L3: pretty large, on bus
        \end{itemize}
    \end{itemize}
\end{itemize}

Traps
\begin{itemize}
    \item Traps: special conditions detected by architecture
    \item On detecting a trap, the hardware
    \begin{itemize}
        \item Saves the state of the process
        \item Transfers control to appropriate trap handler
        \begin{itemize}
            \item The CPU indexes the memory-mapped trap vector with the trap number,
            \item then jumps to the address given in the vector, and
            \item starts to execute at that address
            \item On completion, the OS resumes execution of the process
        \end{itemize}
    \end{itemize}
\end{itemize}

I/O Control
\begin{itemize}
    \item Each I/O device has a little processor that enable autonomous function
    \item CPU issues commands to I/O devices 
    \item When an I/O device complete, it issues an interrupt
    \item CPU stops and processes the interrupt
\end{itemize}

Memory Mapped I/O
\begin{itemize}
    \item Enable direct access to I/O controller
    \item PCs reserve a part of the memory and put the device manager in that memory
    \item Access becomes very fast
\end{itemize}

Interrupt-based asynchronous I/O
\begin{itemize}
    \item Device controller has its own small processor
    \item Puts an interrupt on the bus when done
    \item CPU gets interrupt
    \begin{itemize}
        \item Save critical state
        \item Disable interrupts
        \item Save state that interrupt handler will modify
        \item Invoke interrupt handler using the interrupt vector
        \item Restore software state
        \item Enable interrupts
        \item Restore hardware state
    \end{itemize}
\end{itemize}
\newpage
\section{Process API and Abstraction}

OS provides process abstraction
\begin{itemize}
    \item When you run an exe file, the OS creates a process = a running program
    \item OS timeshares the CPU across multiple processes: virtualizes CPU
    \begin{itemize}
        \item No. of running processes > No. of physical CPU cores
        \item Programs are responsive to users, even when the CPU is busy
    \end{itemize}
    \item OS has a CPU scheduler that picks one of the active processes to execute
    \begin{itemize}
        \item Policy: which to run
        \item Mechanism: how to context switch
    \end{itemize}
\end{itemize}

What constitutes a process?
\begin{itemize}
    \item PID
    \item Memory image
    \begin{itemize}
        \item Code and data (static)
        \item Stack and heap (dynamic)
    \end{itemize}
    \item CPU context: registers
    \begin{itemize}
        \item Program counter
        \item Current operands
        \item Stack pointer
    \end{itemize}
    \item File descriptors
    \begin{itemize}
        \item Pointers to open files and devices
        \item e.g. STDOUT, STDIN, STDERR
    \end{itemize}
\end{itemize}

States of a process
\begin{itemize}
    \item Running: executing
    \item Ready: waiting
    \item Blocked: not ready to run
    \begin{itemize}
        \item Why? Waiting for something
        \item Unblocked when disk issues an interrupt
    \end{itemize}
    \item New: being created
\end{itemize}

OS data structures
\begin{itemize}
    \item OS maintains a data structure of all active processes
    \item Information about each process is stored in a process control block
    \begin{itemize}
        \item Identifier
        \item State
        \item Pointers to related processors
        \item CPU context
        \item Memory pointers
        \item Pointers to open files
    \end{itemize}
\end{itemize}

What API does the OS provide to user programs?
\begin{itemize}
    \item API
    \item Set of system calls
    \begin{itemize}
        \item System call is a function call into the OS code that runs at a higher privilege
        \item Sensitive operations are allowed only at this high privilege
        \item Some blocking system calls cause the process to be blocked and descheduled
    \end{itemize}
\end{itemize}

Should we rewrite programs for each OS?
\begin{itemize}
    \item POSIX API: a standard set of system calls that an OS must implement
    \begin{itemize}
        \item POSIX: Portable Operating System Interface
        \item Most modern OSes are POSIX compliant
    \end{itemize}
    \item Program language libraries hide the system calls
\end{itemize}

Process related system calls in Unix
\begin{itemize}
    \item fork() creates a child process
    \item exec() makes a process execute a given executable
    \item exit() terminates a process
    \item wait() causes a parent to block until child terminates
    \item Many variants exist with different arguments
\end{itemize}

What happens during a fork?
\begin{itemize}
    \item A new process is created by copying the parent's memory image
    \item The new process is added to process list and schedules
    \item Then the parent and child operate independently
    \item On success, the PID of the child process is returned in the parent
    \begin{itemize}
        \item 0 is returned in the child
        \item On failure, -1 is returned to the parent and no child process is created
    \end{itemize}
\end{itemize}

What happens during exec?
\begin{itemize}
    \item After fork, parent and child are running same code
    
    Not too useful!
    
    \item A process can run exec() to load a new process 
\end{itemize}

How does a shell work?
\begin{itemize}
    \item In a basic OS, the init process is created after initialization of hardware
    \item The init process spawns a shell like bash
    \item Shell reads user command, forks a child, execs the command executable, waits, and reads
    \item Common commands like ls are all executables that are simply exec'd by the shell
\end{itemize}
\newpage
\section{Proper Mechanism of Execution}

Low-level mechanisms
\begin{itemize}
    \item How does the OS run a process?
    \item How does it handle a system call?
    \item How does it context switch?
\end{itemize}

Process Execution
\begin{itemize}
    \item OS allocates memory and creates memory image
    \item Points CPU program counter to current instruction
    \item After setup, OS steps out of the way
\end{itemize}

A simple function call
\begin{itemize}
    \item A function call translates to a jump instruction
    \item A new stack frame is pushed to stack and stack pointer is updated
    \item Old value of PC is saved to stack frame and PC is updated
    \item Stack frame contains return value, function arguments etc.
\end{itemize}

How is a system call different?
\begin{itemize}
    \item System calls are in kernel memory space
    \item Kernel uses a separate stack from the user stack
    \item Kernel does not trust user provided addresses
    \item Instead, it sets up an Interrupt Descriptor Table to keep track
\end{itemize}

Mechanism of system call: trap instruction
\begin{itemize}
    \item When a system call must be made, a special trap instruction is run
    \item Trap execution
    \begin{itemize}
        \item Move CPU to higher privilege level
        \item Switch to kernel stack
        \item Save context on kernel stack
        \item Look up address in IDT and jump to trap handler function in OS code
    \end{itemize}
\end{itemize}

More on the trap instuction
\begin{itemize}
    \item Executed on hardware in following cases:
    \begin{itemize}
        \item System call (program needs OS service)
        \item Program fault (illegal stuff)
        \item Interrupt (device needs attention)
    \end{itemize}
    \item Across all cases, the mechanism is the same
    \item IDT has many entries: which to use?
    \item System calls and interrupts store a number in a CPU register before calling trap, as an IDT index
\end{itemize}

Return from trap 
\begin{itemize}
    \item When OS is done handling, it calls a special return-from-trap instruction 
    \begin{itemize}
        \item Restore context
        \item Lower privilege
        \item Restore PC
    \end{itemize}
    \item OS doesn't always return to the same process
\end{itemize}

Why switch?
\begin{itemize}
    \item Sometimes when OS is in kernel mode, it cannot return back to the same process because of terminating or a blocking system call
    \item Sometimes it doesn't want to (timeshare)
    \item In such cases, a context switch is done
\end{itemize}

The OS scheduler
\begin{itemize}
    \item Two parts
    \begin{itemize}
        \item Scheduler which picks processes
        \item Mechanism to switch to that process
    \end{itemize}
    \item Non preemptive schedulers are polite (only if process blocked or terminated)
    \item Preemptive schedulers can switch even when a process is ready to continue
    \begin{itemize}
        \item CPU generates periodic timer interrupts
        \item After servicing, the OS checks if the process has been running too long
    \end{itemize}
\end{itemize}
\newpage
\section{Scheduling Policies}

What are we trying to optimize?
\begin{itemize}
    \item Maximize utilization
    \item Minimize average turnaround time
    \item Minimize average resopnse time
    \item Fairness
    \item Minimize overhead 
\end{itemize}

FIFO
\begin{itemize}
    \item Example: three processes arrive at t=0 in the order A,B,C
    \item Schedule A, then B, then C
    \item Problem: convoy effect (if the first process is long, turnaround times go high)
\end{itemize}

SJF (Shortest Job First)
\begin{itemize}
    \item Provably optimal when all processes arrive together
    \item SJF is non-preemptive, so short jobs can still get stuck when short jobs arrive late
\end{itemize}

STCF (Shortest Time-to-Completion)
\begin{itemize}
    \item Pre-emptive scheduler
    \item Pre-empts running task if time left is more than that of new arrival 
\end{itemize}

RR (Round Robin)
\begin{itemize}
    \item Every process executes for a fixed quantum slice
    \item Slice big enough to amortize cost of context switch
    \item Pre-emptive
    \item Good for response time and fairness
    \item Bad for turnaround time 
\end{itemize}

Schedulers in real systems
\begin{itemize}
    \item Real schedules are more complex
    \item For example, Linux uses a Multi Level Feedback Queue (MLFQ)
    \begin{itemize}
        \item Many queues, with a priority order
        \item Process from highest priority queue scheduled first
        \item Within same priority, any algorithm like RR
        \item Priority of process decays with its age
    \end{itemize}
\end{itemize}
\newpage
\section{Introduction to Virtual Memory}

Why virtualize?
\begin{itemize}
    \item It's messy!
    \item Earlier, memory only had code of one running processes
    \item Now, multiple active processes timeshare
    \item Need to hide this complexity from user
\end{itemize}

Abstration: Virtual Address space
\begin{itemize}
    \item Virtual address space: every process assumes it has maximum memory space
    \item ontains program code, heap, and stack
    \item CPU issues loads and stores to virtual addresses
\end{itemize}

How is actual memory reached?
\begin{itemize}
    \item Address translation from VA to PA
    \item OS allocates memory and tracks loation of processes
    \item Actual translation by MMU
\end{itemize}

Example: Paging
\begin{itemize}
    \item OS divides virtual address space into fixed size pages, physial memory into frames
    \item To allocate memory, a page is mapped to a free physical frame
    \item Page table stores mappings from virtual page to physical frame
    \item MMU has access to page tables
\end{itemize}

Goals of memory virtualization
\begin{itemize}
    \item Transprency: user programs shouldnt handle the details
    \item Effiiency: minimize overhead
    \item Isolation and protection: a user program should not be able to access anything outside its address space
\end{itemize}

How can a user allocate memory?
\begin{itemize}
    \item OS allocates a set of pags to the memory image of the process
    \item Within this image, the process can allocate memory
\end{itemize}
\newpage
\section{Demand Paging}

Main memory is not always enough

We have x amount of frames in physical memory, but n amount of frames in swap space.

Swap these on demand

Present bit in PTE, indicates if a page is in memory or not.

If present bit is not set on MMU lookup, page fault

OS moves to kernel mode, switches to another process while lookup happens

What happens on memory access:
\begin{itemize}
    \item CPU issuses load to VA for data
    \item Checks all 3 caches
    \item In case of miss, goes to main memory
    \item MMU issues request to TLB
    \item If no hit, page fault
\end{itemize}

Complications during a page fault:
\begin{itemize}
    \item If no room in memory, must issue a swap.
    \item Time consuming
\end{itemize}

Replacement policies: FIFO, LRU

\section{LRU Approximation}

Second chance algorithm:

Maintain a reference bit next to each frame in a circular linked list

If the reference bit is 1, set it to 0 and move on

If it is 0, replace

If we use the page, set the bit to 1

Enhanced second chance, two bits

\section{Stuff I didn't write down yet}


\section{Locks}

Locks: Basic idea

When updating some shared variable, we can use a lock variable to protect it

\begin{itemize}
    \item Mutual exclusion
    \item Fairness
    \item Low overhead
\end{itemize}

Naive implementation:

Lock disables interrupts

Unlock enables them

Flaws:
\begin{itemize}
    \item Disable and enable is privileged and can be misused
    \item Doesn't work for multi-processor
\end{itemize}

Next implementation:

Lock spins until a flag is 0, then updates it to 1 after

Unlock sets to 0

Flaws:
\begin{itemize}
    \item Busily waiting, takes CPU to do nothing
    \item Still has race conditions (FAILED)
\end{itemize}

Software implementation: almost impossible

Modern architecture uses hardware atomic instructions

Atomic instruction: testAndSet

Delivers mutual exclusion and fairness

Alternative to spinning: yield, gives up CPU

\section{Condition Variables}

Another type of synchronization

Locks allow mutual exclusion

Another common requirement is waiting and signaling

Inefficient to busy-wait, we need a new synchronization primitive.

\section{Semaphores}

A primitive synchronization variable with an underlying counter

Up/post increments and wakes up a blocked process

Down/wait decrements and blocks if negative

A semaphore initialized with n allows n threads

\section{Concurrency Bugs}

Threaded program bugs are non-deterministic

Deadlock bugs: threads cannot execute

Non-deadlock: threads execute with error

Atomicity bugs: assumptions made are violated during execution (solution: locks for mutex)

Order-violation: desired order of memory accesses is flipped (solution: condition variables)
\newpage
\section{Communication with I/O Devices}

I/O devices connect via a bus

High speed: PCI

Others: SCSI, USB, SATA

Simple Device Model:

Block devices store a set number of blocks

Character devices produce/consume streams of bytes

Devices expose an interface of memory registers

Internals usually hidden

OS talks to I/O registers through drivers, but mainly in and out instructions

Also uses memory mapped I/O
\end{document}