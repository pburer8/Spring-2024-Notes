\documentclass[12pt]{article}

\usepackage{amsmath, amsfonts, amssymb, setspace, blindtext, hyperref, tikz, graphicx} \doublespacing

\hypersetup{colorlinks=true,linkcolor=blue,filecolor=magenta,urlcolor=cyan,}
\title{Operating Systems Notes}

\author{Marie Burer}

\date{16-1-24}

\begin{document}

\maketitle
\newpage
\setlength{\parindent}{0pt}

\tableofcontents

\newpage

\section*{Course Outline}

Information:
\begin{itemize}
    \item CS:3620 Operating Systems
    \item Instructor: Guanpeng Li
    \item Location: 110 MacLean Hall
    \item Hours: Tuesday/Thursday, 8:00 AM - 9:15 AM
    \item Prerequesite: CS:2210, CS:2230, and CS:2630 with a minimum grade of C-
    \item TA: A K M Muhitul Islam
\end{itemize}

Modality:
\begin{itemize}
    \item Lectures in person
    \item Office Hours: Tuesday/Thursday, 2:30 PM - 4:30 PM
    \item Location: IATL 340
\end{itemize}

Grading:
\begin{itemize}
    \item Homework: 40\%
    \begin{itemize}
        \item 6-8 assignments
        \item Programming tasks
    \end{itemize}

    \item In-class activities: 20\%
    \begin{itemize}
        \item Quizzes:
        \begin{itemize}
            \item Multiple-choice
            \item Close book
        \end{itemize}
        \item Exercises:
        \begin{itemize}
            \item Open questions
            \item Individual or group
            \item BYO laptop
        \end{itemize}
    \end{itemize}
    
    \item Final Exam: 40\%
    \begin{itemize}
        \item Technical interview
        \item Written exam
    \end{itemize}
\end{itemize}

Policy:
\begin{itemize}
    \item 3 days late policy
    \item Cannot make up quizzes or exercises
    \item Can discuss homework with colleagues
    \item Cannot share code
    \item Cannot copy from internet
    \item If you copy code from StackOverflow, credit it 
    \item When in doubt, ask the instructor/TA
\end{itemize}

\newpage

\section{Introduction to OS}
What is an operating system?

An operating system (OS) is system software that manages computer hardware and software resources and provides common services for computer programs. (Wikipedia)
\begin{itemize}
    \item Definition has changed over years
    \begin{itemize}
        \item Originally, very bare bones
        \item Evolved to include more
    \end{itemize}
    \item Operating System (OS)
    \begin{itemize}
        \item Interface between the user and the architecture
        \item Implements a virtual machine that is easier to program than raw hardware
    \end{itemize}
    \item Middleware between user programs and system hardware
    \item Manages hardware: CPU, main memory, I/O devices
\end{itemize}
OS: Traditional View:
\begin{itemize}
    \item Interface between user and architecture
    \begin{itemize}
        \item Hides architectural details
    \end{itemize}
    \item Implements Virtual machine:
    \begin{itemize}
        \item Easier to program
    \end{itemize}
    \item Illusionist
    \begin{itemize}
        \item Bigger, faster, reliable
    \end{itemize}
    \item Government
    \begin{itemize}
        \item Divides resources
        \item "Taxes" = overhead
    \end{itemize}
\end{itemize}
New Developments in OS:
\begin{itemize}
    \item Operating systems: active field of research
    \begin{itemize}
        \item Demands on OS's growing
        \item New application spaces (Web, Grid, Cloud)
        \item Rapidly evolving hardware
    \end{itemize}
    \item Open-source operating systems
    \begin{itemize}
        \item Linux etc.
        \item You can contribute to and develop OS's
        \item Excellent research platform
    \end{itemize}
\end{itemize}
OS: Salient Features
\begin{itemize}
    \item Services: The OS provides standard services which the hardware implements
    \item Coordination: The OS coordinates multiple applications and users to achieve fairness and efficiency
    \item Design an OS so that the machine is convenient to use and efficient
\end{itemize}
Why study OS?
\begin{itemize}
    \item Abstraction: How to get the OS to give users an illusion of infinite resources
    \item System Design: How to make tradeoffs between performance, convenience, simplicity, and functionality
    \item Basic Understanding: The OS provides the services that allow application programs to work at all
    \item System Intersection Point: The OS is the point where hardware and software meet
\end{itemize}

Not many operating systems are under development, so you are unlikely to get a job building an OS. However, understanding operating systems will enable you to use your computer more efficiently \newline
Build Large Computer Systems
\begin{itemize}
    \item OS as an example of large system design
    \item Goals: 
\end{itemize}

\section{OS Structure and Architecture}

Modern OS Functionality:
\begin{itemize}
    \item Concurrency
    \item I/O Devices
    \item Memory management
    \item Files
    \item Distributed systems and networks
\end{itemize}

OS Principles:
\begin{itemize}
    \item OS as juggler
    \item OS as government
    \item OS as complex system
    \item OS as history teacher
\end{itemize}

Protection:
\begin{itemize}
    \item Kernel mode vs User Mode
    
    To protect the system from aberrant users, some instructions are restricted to use only by the OS

    Users may not
    \begin{itemize}
        \item address I/O directly
        \item use instructions that manipulate the state of memory
        \item set the mode bits that determine user or kernel mode
        \item disable and enable interrupts
        \item halt the machine
    \end{itemize}

    In kernel mode, the OS can do all of these things

    \item Hardware needs to support at least both kernel and user
\end{itemize}

Crossing Protection Boundaries:

System call: OS procedure that executes privileged instructions
\begin{itemize}
    \item Causes a trap
    \item The trap handler uses the parameter to jump
    \item The handler saves caller's state so it can restore control
    \item The architecture must permit this
\end{itemize}

Memory Protection:
\begin{itemize}
    \item Architecture must provide support to the OS to protect user programs and itself
    \item The simplest technique is to use base and limit registers
    \item Instantiated by the OS before starting a program
    \item The CPU checks user reference ensuring it falls between base and limit values
\end{itemize}

Memory Hierarcy:

Higher = small, fast, more \$, less latency
\begin{itemize}
    \item registers (1 cycle)
    \item L1 (2 cycle)
    \item L2 (7 cycle)
    \item RAM (100 cycle)
    \item Disk (40,000,000 cycle)
    \item Network (200,000,000+ cycle)
\end{itemize}

Process Layout in Memory
\begin{itemize}
    \item Stack
    \item Gap 
    \item Data 
    \item Text
\end{itemize}

Caches
\begin{itemize}
    \item Access to main memory is expensive
    \item Caches are small, fast, inexpensive
    \begin{itemize}
        \item Different sizes and locations:
        \begin{itemize}
            \item L1: on-chip, smallish
            \item L2: next to chip, larger
            \item L3: pretty large, on bus
        \end{itemize}
    \end{itemize}
\end{itemize}

Traps
\begin{itemize}
    \item Traps: special conditions detected by architecture
    \item On detecting a trap, the hardware
    \begin{itemize}
        \item Saves the state of the process
        \item Transfers control to appropriate trap handler
        \begin{itemize}
            \item The CPU indexes the memory-mapped trap vector with the trap number,
            \item then jumps to the address given in the vector, and
            \item starts to execute at that address
            \item On completion, the OS resumes execution of the process
        \end{itemize}
    \end{itemize}
\end{itemize}

I/O Control
\begin{itemize}
    \item Each I/O device has a little processor that enable autonomous function
    \item CPU issues commands to I/O devices 
    \item When an I/O device complete, it issues an interrupt
    \item CPU stops and processes the interrupt
\end{itemize}

Memory Mapped I/O
\begin{itemize}
    \item Enable direct access to I/O controller
    \item PCs reserve a part of the memory and put the device manager in that memory
    \item Access becomes very fast
\end{itemize}

Interrupt-based asynchronous I/O
\begin{itemize}
    \item Device controller has its own small processor
    \item Puts an interrupt on the bus when done
    \item CPU gets interrupt
    \begin{itemize}
        \item Save critical state
        \item Disable interrupts
        \item Save state that interrupt handler will modify
        \item Invoke interrupt handler using the interrupt vector
        \item Restore software state
        \item Enable interrupts
        \item Restore hardware state
    \end{itemize}
\end{itemize}
\end{document}