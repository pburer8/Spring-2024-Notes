\documentclass[12pt]{article}

\usepackage{amsmath, amsfonts, amssymb, setspace, blindtext, hyperref, tikz, graphicx} \doublespacing

\hypersetup{colorlinks=true,linkcolor=blue,filecolor=magenta,urlcolor=cyan,}
\title{Intermediate Microeconomics: Advanced Notes}

\author{Marie Burer}

\date{16-1-24}

\begin{document}

\maketitle
\newpage
\setlength{\parindent}{0pt}

\tableofcontents

\newpage

\section{Mathematics of Optimization}

\subsection{Unconstrained Optimization}

\begin{itemize}
    \item First order conditions: $\frac{\delta f}{\delta x} = 0, \frac{\delta f}{\delta y} = 0$
    \item Second order conditions: $\frac{\delta^{2} f}{\delta x^{2}} < 0, \frac{\delta^{2} f}{\delta y^{2}} < 0$
    \item Additionally, $(\frac{\delta^{2} f}{\delta x^{2}} \times \frac{\delta^{2} f}{\delta y^{2}}) - (\frac{\delta^{2} f}{\delta x \delta y})^{2} > 0$
\end{itemize}

Example: $G(T,C) = 50 + 10T + 16C - T^{2} - 2TC - 2C^{2}$

$\frac{\delta G}{\delta T} = 10 - 2T - 2C$

$\frac{\delta G}{\delta C} = 16 - 2T - 4C$

$\begin{bmatrix}
    -2 & -2 & -10 \\
    -2 & -4 & -16
\end{bmatrix} \Rightarrow \begin{bmatrix}
    -2 & -2 & -10 \\
    0 & -2 & -6
\end{bmatrix} \Rightarrow \begin{bmatrix}
    -2 & 0 & -4 \\
    0 & -2 & -6
\end{bmatrix}$

$\Rightarrow T = 2, C = 3$

$\frac{\delta^{2} G}{\delta T^{2}} = -2 < 0$

$\frac{\delta^{2} G}{\delta C^{2}} = -4 < 0$

$(\frac{\delta^{2}G}{\delta T^{2}} \times \frac{\delta^{2} G}{\delta C^{2}}) - (\frac{\delta^{2}G}{\delta T \delta C})^{2} = (-2 \times -4) - (-2)^{2} = 8 - 4 = 4 > 0$

$\Rightarrow G(2, 3) \text{ is a maximum } \Rightarrow G(2,3) = 50 + 20 + 48 - 4 - 12 - 18 = 84$

\subsection{Constrained Optimization}

To find the critical points of $f(x,y)$ subject to some constraints $c - g(x,y) = 0$ we have

Lagrangian: $L = f(x,y) - \lambda(c - g(x,y))$

We take the derivative wrt x, y, and $\lambda$ to get

$\begin{bmatrix}
    \frac{\delta f}{\delta x} & = \lambda\frac{\delta g}{\delta x} \\
    \frac{\delta f}{\delta y} & = \lambda\frac{\delta g}{\delta y} \\
    g(x) & = c
\end{bmatrix}$

Example: $T = 19.25 - 6\ln(R) - 4\ln(W)$, constraint is $R + W = 5$

$\begin{bmatrix}
    19.25 - \frac{6}{R} & = \lambda \\
    19.25 - \frac{4}{W} & = \lambda \\
    R + W & = 5
\end{bmatrix}$

$19.25 - \frac{6}{R} = 19.25 - \frac{4}{W} \Rightarrow 6W = 4R$

$W = 5 - R \Rightarrow 6(5 - R) = 4R \Rightarrow 30 - 6R = 4R \Rightarrow 10R = 30 \Rightarrow R = 3, W = 2$

$T(3, 2) = 19.25 - 6\ln(3) - 4\ln(2) \approx 9.89$

Example: $I = \frac{P^{2}}{250} + \frac{PA}{100} + \frac{A^{2}}{1000}$, constraint is $P + A = 110$

$\begin{bmatrix}
    \frac{P}{125} + \frac{A}{100} & = \lambda \\
    \frac{P}{100} + \frac{A}{500} & = \lambda \\
    P + A & = 110
\end{bmatrix}$

$\frac{P}{125} + \frac{A}{100} = \frac{P}{100} + \frac{A}{500}$

$\frac{P}{125} - \frac{P}{100} = \frac{-4A}{500}$

$\frac{-P}{500} = \frac{-4A}{500}$

$P = 4A \Rightarrow 5A = 110 \Rightarrow A = 22 \Rightarrow P = 88$

$I(88, 22) = 50.82$

\section{Consumer Theory}

Defining preferences:
\begin{itemize}
    \item Preferences reflect choices
    \begin{itemize}
        \item Defined over bundles of goods
        \item Based upon all properties of a bundle
    \end{itemize}
    \item Notation
    \begin{itemize}
        \item Strictly Preferred $\succ$
        \item Weakly Preferred $\succeq$
        \item Indifferent $\sim$
    \end{itemize}
    \item Preferences are assumed to be well-bheaved
    \begin{itemize}
        \item Complete: An individual knows hiw or her preferences over any two bundles
        \item Reflexive: An individual must be indifferent between two identical bundles of goods
        \item Transitivity: No cycles
    \end{itemize}
\end{itemize}

Anchoring and Adjustment:

In multi-dimensional problems, people tend to anchor on one dimension and adjust for the other.

\subsection{Utility Functions}

A utility function is a function from bundles of goods into real numbers.
\begin{itemize}
    \item $x \succ y \Rightarrow u(x) > u(y)$
    \item More preferred bundles must be assigned larger numbers
    \item If preferences are well-behaved, we can write down a utility function that captures the preferences
\end{itemize}

Ordinal utility function: the only thing that matters is the order

Cardinal utility function: the numbers have significance and meaning

Assumptions about utility functions:
\begin{itemize}
    \item Monotonicity: $x > x' \Rightarrow u(x,y) \ge u(x', y)$
    \item Convexity: Let $\alpha \text{ s.t. } 0 < \alpha < 1$. If $u(x,y) = u(x',y'), u(\alpha x + (1 - \alpha)x', \alpha y + (1 - \alpha)y') \ge u(x,y)$
\end{itemize}

Indifference curve: a curve along which $u(x,y) = c$ for some constant c

Marginal Utility: the marginal utility of x ($MU_{x}$) is defined as the increase in utility from consuming one additional good of x

$MU_{x} = \frac{\delta u}{\delta x}$

The marginal rate of substitution of x, y is the amount of good y needed to make up for the loss of one of good x

$MRS_{xy} = \frac{MU_{x}}{MU_{y}}$

\subsection{Budget Constraints}

The budget line is the line along which the amount spent exactly equals the available budget.

Let B be the budget, and $p_{x}, p_{y}$ be the price of good $x$ or good $y$. Thus, we have the budget line as $p_{x}x + p_{y}y = B$

Solving for $y$, we see have $y = \frac{B}{p_{y}} - \frac{p_{x}}{p_{y}}x$

The slope of the budget line is $\frac{p_{x}}{p_{y}}$, which can be interpreted as the opportunity cost of good $x$ in terms of good $y$

\subsubsection{Non-linear Constraints}

Complexities
\begin{itemize}
    \item Charity with restrictions
    \item Taxes with brackets
    \item Programs with income eligibility
\end{itemize}

\subsection{Conditions for Utility Maximization}

Consider an individual choosing how much they want to consume, they are choosing optimally, with a positive amount of all goods.

Then, the following must be true at their optimal consumption bundle: $MRS_{xy} = \frac{p_{x}}{p_{y}}$
\end{document}