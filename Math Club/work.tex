\documentclass[12pt]{article}

\usepackage{amsmath, amsfonts, amssymb, setspace, blindtext, hyperref, tikz, graphicx} \doublespacing

\hypersetup{colorlinks=true,linkcolor=blue,filecolor=magenta,urlcolor=cyan,}

\begin{document}

\setlength{\parindent}{0pt}

Consider the notation:

3 $\uparrow$ 3 = $3 \times 3 \times 3$, as in three 3's multiplied together

3 $\uparrow\uparrow$ 3 = $3^{3^{3}}$, as in three 3's in a tower of exponentation

In general, 3 $\uparrow_{n}$ 3 = $3 \uparrow_{n-1} 3 \uparrow_{n-1} 3$

Now consider, $g_{1} = 3 \uparrow_{4} 3 \approx 1.25 \times 10^{3638334640024}$

Let $g_{n} = 3 \uparrow_{g_{n-1}} 3$

Graham's number is defined as $g_{64}$

If you tried to imagine every digit of Graham's number, your brain would collapse into a black hole!
\end{document}